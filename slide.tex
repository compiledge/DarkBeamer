% \documentclass[dracula,12pt]{darkbeamer}
% \documentclass[aspectratio=169,tokyonightday,Frankfurt,12pt]{darkbeamer}
% \documentclass[aspectratio=169,solarizeddark,Frankfurt,12pt]{darkbeamer}
% \documentclass[aspectratio=169,solarizedlight,Frankfurt,12pt]{darkbeamer}
% \documentclass[aspectratio=169,tokyonight-storm,Frankfurt,12pt]{darkbeamer}
% \documentclass[aspectratio=169,dracula,Frankfurt, 90pt]{darkbeamer}
\documentclass[aspectratio=169,tokyonight-moon,Frankfurt, 90pt]{darkbeamer}
% \documentclass[aspectratio=169,gruvbox-light,Frankfurt, 90pt]{darkbeamer}


\author{author}
\title{title}
\subtitle{subtitle}
\institute{My University}

\begin{document}

\begin{frame}{}
	\titlepage
\end{frame}

\begin{frame}{Summary}
	\label{contents}
	\tableofcontents
\end{frame}

\section{Basic Formating}
\subsection{Emphasis Text and Columns}
\begin{frame}{Emphasis Text and Columns}
	Text Effects:
	\begin{columns}[c]
		\column{.45\textwidth}
		\begin{outline}
			\1 \emph{Emphasis text}
			\1 \textbf{Bold text}
			\1 \textit{Italic text}
			\1 \texttt{Teletype text}
			\1 \alert{Alert text}
			\1 \textrm{Roman font text}
		\end{outline}
		\column{.45\textwidth}
		\begin{outline}[enumerate]
			\1 Vim:
			\2 Neovim
			\2 Astrovim
			\1 emacs
		\end{outline}
		\begin{outline}
			\1 Custom itens:
			\2[\gd] Good
			\2[\bd] Bad
		\end{outline}
	\end{columns}
\end{frame}

\begin{frame}{Menu and Keys}
    Menu keys:
    \begin{outline}
        \1 Menu: \menu{nvim > telescope}
		  \1 Keys: \keys{\ctrl + c}, \keys{\ctrl +v}
        \1 Filepath \directory{.dot/.config/nvim} 
    \end{outline}
\end{frame}

\subsection{Images}
\subsection{Tables}

\section{Blocks}
\subsection{Beamer Blocks}

\begin{frame}[allowframebreaks]{Beamer Blocks}
	\begin{block}{Standard block title}
		hello world!
		\begin{outline}
			\1 Item
			\1 Item
		\end{outline}

	\end{block}

	\begin{alertblock}{Alert block title}
		hello world!
	\end{alertblock}

	\begin{exampleblock}{Example block title}
		hello world!
	\end{exampleblock}
\end{frame}

\begin{frame}[allowframebreaks]{Theorem Environments}

	\begin{theorem}
		Hello World!
	\end{theorem}

	\begin{corollary}
		Hello World!
	\end{corollary}

	\begin{definition}
		Hello World!
	\end{definition}

	\begin{definitions}
		Hello World!
	\end{definitions}

	\begin{fact}
		Hello World!
	\end{fact}

	\begin{lemma}
		Hello World!
	\end{lemma}

	\begin{example}
		Hello World!
	\end{example}

	\begin{examples}
		Hello World!
	\end{examples}

\end{frame}


\subsection{Tcolorbox Blocks}

\begin{frame}[allowframebreaks]{Tcolorbox Blocks}
	\begin{tbox}{Text box}
		Hello world!
	\end{tbox}

	\begin{gbox}{Good box}
		Hello world!
	\end{gbox}

	\begin{wbox}{Warning box}
		Hello world!
	\end{wbox}

	\begin{bbox}{Bad box}
		Hello world!
	\end{bbox}

	\begin{lbox}{Link box}
		Hello world!
	\end{lbox}

	\begin{cbox}{Code box}
		Hello world!
	\end{cbox}

	\begin{forkbox}{Fork box}
		Hello world!
	\end{forkbox}

	\begin{githbox}{Github box}
		Hello world!
	\end{githbox}

	\begin{gitlbox}{Gitlab box}
		Hello world!
	\end{gitlbox}
\end{frame}

\section{Snippets}
\label{snippets}
\subsection{Code Snippets}

\begin{frame}[fragile]{Code Snippets}
\begin{lstlisting}[language=c, caption=C snippet]
#include<stdio.h>
int main(){
	printf("Hello World!\n");
	return 0;
}//end[main]
\end{lstlisting}
\end{frame}

\subsection{Console snippets}

\begin{frame}[fragile]{Code Snippets}
\begin{lstlisting}[language=bash, caption=Debian snippet]
$ apt install vim 
\end{lstlisting}

\begin{lstlisting}[language=python, caption=Python snippet]
>>> print "hello world"
\end{lstlisting}
\end{frame}

\subsection{No Starch Press Style}

\begin{frame}[fragile]{No Starch Press Style}
\begin{lstlisting}[language=c, caption=C language example]
#include<stdio.h>

int main(){

(*@\lnnum{6}@*)  return 0;
(*@\lnnum{1}@*)  printf("Hello World\n");
}
\end{lstlisting}
The \lnnum{6} item should be the last instruction!
\end{frame}

\section{Hyperlinks \& Buttons}

\begin{frame}{Hyperlinks \& Buttons}
	More options can be found in \href{www.duckduckgo.com}{Duck Duck Go}, {\it i.e.}, \url{www.duckduckgo.com}.
\begin{outline}
	\1 Contenst page:
		\2 \hyperlink{contents}{\beamerbutton{contents page}}
	\1 Blocks section:
		\2 \hyperlink{blocks}{\beamergotobutton{columns section}}
		\2 \hyperlink{blocks}{\beamerskipbutton{block section}}
		\2 \hyperlink{blocks}{\beamerreturnbutton{block section}}
	\1 Snippets section:
		\2 \hyperlink{snippets}{\beamergotobutton{code section}}
		\2 \hyperlink{snippets}{\beamerskipbutton{code section}}
		\2 \hyperlink{snippets}{\beamerreturnbutton{code section}}
\end{outline}

\end{frame}

\end{document}
